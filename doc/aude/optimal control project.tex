\documentclass{article}
\begin{document}

1. Let's formulate this problem as an optimal control problem. \\

Define the state equations : \\

$\dot{X} = V_{x} \cdot cos(\psi) -V_{y} \cdot sin(\psi) $\\

$\dot{Y} = V_{x} \cdot sin(\psi) + V_{y} \cdot cos(\psi)$ \\

$\dot{V_{x}} = \frac{1}{m} \cdot (f_{F_{x}} \cdot cos(\delta) - f_{F_{y}} \cdot sin(\delta) + f_{R_{x}}) - V_{} \cdot r $\\

$\dot{V_{y}} = \frac{1}{m} \cdot (f_{F_{x}} \cdot sin(\delta) + f_{F_{y}} \cdot cos(\delta) + f_{R_{y}}) - V_{x} \cdot r $\\

$\dot{r} = \frac{1}{I_{z}} \cdot (f_{F_{y}} \cdot cos(\delta) + f_{F_{x}} \cdot sin(\delta)) \cdot l_{F} - f_{R_{y}} \cdot l_{R}$\\

$\dot{\psi} = r $\\


Define the inputs : \\

$f_{F_{x}} , f_{R_{x}} , \delta$ \\

Define the performance measure : \\

$J = \alpha \cdot \phi(x(t_{f}),t_{f}) + (1 - \alpha ) \cdot \int_0^{t_{f}} L(x,u,t) dt$\\

$J = \alpha \cdot\ y(t_{f})^2 + t_{f} = \alpha \cdot\ y(t_{f})^2 + (1-\alpha) \cdot \int_0^{t_{f}} 1  dt$ \\

The boundary conditions are : \\

$\beta(0) = 0$\\

$V_{x}(0) = 55 km per hour $\\

$V_{y}(0) = 0$\\

$\beta(t_{f}) \simeq - 90 degrees$\\

$\dot{\psi}(t_{f}) = 0$\\

The equations given are the following :\\

$\beta = arctan(\frac{V_{y}}{V_{x}} = arctan(\frac{\dot{X}}{\dot{Y}}) - \psi$\\

$f_{\star y} = f^{max}_{\star y} \cdot sin(C \cdot arctan(B \cdot s_{\star y})) \quad \star = F,R$\\

$s_{F_{y}} = \frac{V \cdot sin(\beta - \delta) + r \cdot l_{f} \cdot cos(\delta)}{V \cdot cos(\beta - \delta) + r \cdot l_{f} \cdot sin(\delta)}$\\

$s_{R_{y}} = \frac{V \cdot sin(\beta) - r \cdot l_{R}}{V \cdot cos(\beta)}$\\

$f^{max}_{\star y} = \sqrt{(\mu f_{\star z})^2 -f^2_{\star x}}, \quad \star = F,R$\\

$f_{F_{z}} = m \cdot g \cdot \frac{l_{R}}{l_{F} + l_{R}}$\\

$f_{R_{z}} = m \cdot g \cdot \frac{l_{F}}{l_{F} + l_{R}}$\\

$|f_{\star x}| \leq \mu \cdot f_{\star z} = f^{max}_{\star x}, \quad \star = F,R $\\


2. Assuming the rear-wheel driven (RWD) vehicle, let us derive the necessary conditions for this optimization problem. \\

For a rear-drive vehicle, 
$f_{F_{x}} = 0$ when driving and $f_{F_{x}} < 0$ when braking, while $f_{R_{x}}$ can take both positive (when driving) and negative (when braking) values.\\
Ideally, $f_{F_{x}} <0$ if and only if $f_{R_{x}} < 0$.\\

Define the Hamiltonian for the optimal-control problem : \\

$H(x,\lambda,u,t) = L(x,u,t) + \lambda^T \cdot f(x,u,t) + \mu^T \cdot C(x,u,t)$\\

$C(x,u,t)$ represents the inputs constraints. \\

$C_{1}(x,u,t) = - f_{F_{x}} \cdot f_{R_{x}} < 0$\\

$C_{2}(x,u,t) = (\sqrt{\delta^2} - \delta_{max}) \leq 0$\\

$C_{3}(x,u,t) = (\sqrt{f^2_{R_{x}}} - f^{max}_{R_{x}}) \leq 0 $\\

$C_{4}(x,u,t) = (\sqrt{f^2_{F_{x}}} - f^{max}_{F_{x}}) \leq 0$\\

$H(x(t),u(t),t) = 1 + \lambda_{x} \cdot V_{x} + \lambda_{y} \cdot V_{y} +\\ \lambda_{V_{x}} \cdot  \frac{1}{m} \cdot [(f_{F_{x}} \cdot cos(\delta) - f_{F_{y}} \cdot sin(\delta) + f_{R_{x}}) - V_{} \cdot r] +\\
 \lambda_{V_{y}} \cdot[ \frac{1}{m} \cdot (f_{F_{x}} \cdot sin(\delta) + f_{F_{y}} \cdot cos(\delta) + f_{R_{y}}) - V_{x} \cdot r] + \\
\lambda_{r} \cdot [ \frac{1}{I_{z}} \cdot (f_{F_{y}} \cdot cos(\delta) + f_{F_{x}} \cdot sin(\delta)) \cdot l_{F} - f_{R_{y}} \cdot l_{R}] + \\
\lambda_{\psi} \cdot r$ + \mu_{1} \cdot(- f_{F_{x}} \cdot f_{R_{x}}) + \mu_{2} \cdot (\sqrt{\delta^2} - \delta_{max}) + \\
\mu_{3} \cdot (\sqrt{f^2_{R_{x}} - f^{max}_{R_{x}}}) +  \mu_{4} \cdot (\sqrt{f^2_{F_{x}} - f^{max}_{F_{x}}})  \\

The co-state equations are the following : \\

$\dot{\lambda_{x}} = - \frac{\partial H}{\partial x} = 0$\\

$\dot{\lambda_{y}} = - \frac{\partial H}{\partial y} = 0$\\

$\dot{\lambda_{V_{x}}} = - \frac{\partial H}{\partial V_{x}} = \lambda_{V_{y}} \cdot r$\\

$\dot{\lambda_{V_{y}}} = - \frac{\partial H}{\partial V_{y}} = - \lambda_{V_{x}} \cdot r$\\

$\dot{\lambda_{r} = - \frac{\partial H}{\partial r} = - \lambda_{V_{x}} \cdot V_{y} + \lambda_{V_{y}}} \cdot V_{x} $\\

$\dot{\lambda_{\psi}} = - \frac{\partial H}{\partial \psi} = 0$\\


3. Let us give the boundary conditions for the states $r(t_{f}) = 0$\\

The transversality condition for the Hamiltonian is : \\

$\frac{\partial \phi}{\partial t}(t_{f},x_{f}) + H(t_{f},x_{f}) = 0$\\

that is $H(t_{f},x_{f}) = -1$\\

The transversality conditions for the co-states are : \\

$\frac{\partial \phi}{\partial X}(t_{f},x_{f}) - \lambda(t_{f}) = 0$\\

that is :\\

$\frac{\partial \phi}{\partial x}(t_{f},x_{f}) = \lambda_{x}(t_{f}) = 0$\\

$\frac{\partial \phi}{\partial y}(t_{f},x_{f})  = \lambda_{y}(t_{f}) = \alpha \cdot 2 \cdot y(t_{f}) $\\

$\frac{\partial \phi}{\partial V_{x}}(t_{f},x_{f}) = \lambda_{V_{x}}(t_{f}) = 0$\\

$\frac{\partial \phi}{\partial V_{y}}(t_{f},x_{f}) = \lambda_{V_{}}(t_{f}) = 0$\\

$\frac{\partial \phi}{\partial r}(t_{f},x_{f}) = \lambda_{r}(t_{f}) = 0$\\

$\frac{\partial \phi}{\partial \psi}(t_{f},x_{f}) = \lambda_{\psi}(t_{f}) = 0$\\

4. Let us determine the optimal control strategy : \\

Looking for a singular control, \\

$\frac{\partial H}{\partial u} = 0$ which is equivalent to three equations \\

$\frac{\partial H}{\partial f_{F_{x}}} = 0$ that is \\

$\frac{\lambda_{V_{x}}}{m} \cdot cos(\delta) + \frac{\lambda_{V_{y}}}{m} \cdot sin(\delta) + \frac{\lambda_{r}}{I_{z}} \cdot sin(\delta) - \mu_{1} \cdot f_{R_{x}}- \mu_{4} \cdot \frac{f_{F_{x}}}{\sqrt{f_{F_{x}}^2}} = 0$\\


$\frac{\partial H}{\partial f_{R_{x}}} = 0$ that is\\ 

$\frac{\lambda_{V_{x}}}{m} - \mu{1} \cdot f_{F_{x}} -\mu_{3} \cdot \frac{f_{R_{x}}}{\sqrt{f_{R_{x}}^2}} = 0$\\

$\frac{\partial H}{\partial \delta}  = 0$ that is \\

$ - \frac{\lambda_{V_{x}}}{m} \cdot f_{F_{x}} \cdot sin(\delta) - \frac{\lambda_{V_{x}}}{m} \cdot f_{F_{y}} \cdot cos(\delta) + \frac{\lambda_{V_{y}}{m}}\cdot f_{F_{x}} \cdot cos(\delta) - \frac{\lambda_{V_{y}}{m}} \cdot f_{F_{y}} \cdot sin(\delta) - \frac{\lambda_{r}}{I_{z}} \cdot f_{R_{y}} \cdot sin(\delta) + \frac{\lambda{r}}{I_{z}} \cdot f_{F_{x}} \cdot cos(\delta) - \mu_{2} \cdot \frac{\delta}{\sqrt{\delta^2}}= 0$ \\

This is not solvable analytically. We tried finding u for higher time derivatives of $H_{u}(x,u,\lambda,\mu,t)$, using the formula :\\

$(-1)^m \cdot \frac{\partial}{\partial u}\cdot (\frac{d^{2m}}{dt^{2m}} \cdot H_{u}(x,u,\lambda,\mu,t)) \geq 0$ , \\

but the equations became very complex and the control still was not appearing explicitly. \\
Therefore we wrote the equations into Mathematica, however this led to even more complex equations. This justifies the fact that we are using GPOPS to try and solve the problem, because the analytical approach is simply too complicated.\\

\end{document}